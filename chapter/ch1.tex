\chapter{Biji Sesawi\\The Mustard Seed}

\bahasa
21 Agustus 1974 pagi di Aula Buddha

\english
21 August 1974 am in Buddha Hall

\bahasa
FIRMAN PERTAMA

\english
THE FIRST SAYING

\bahasa
Murid-murid berkata kepada Yesus: Katakanlah kepada kami, seperti apa kerajaan surga itu?

\english
THE DISCIPLES SAID TO JESUS: TELL US WHAT THE KINGDOM OF HEAVEN IS LIKE.

\bahasa
DIA BERKATA KEPADA MEREKA: KERAJAAN SURGA ITU SEUMPAMA SEBUTIR BIJI SESAWI -- PALING KECIL DARI SEMUA BIJI, TETAPI KETIKA JATUH KE TANAH YANG SUDAH DISIAPKAN BIJI INI MENGHASILKAN SUATU POHON BESAR DAN MENJADI SUATU TEMPAT BERLINDUNG BAGI BURUNG-BURUNG DI SURGA.

\english
HE SAID TO THEM: IT IS LIKE A MUSTARD SEED -- SMALLER THAN ALL SEEDS, BUT WHEN
IT FALLS ON THE TILLED EARTH IT PRODUCES A LARGE TREE AND BECOMES SHELTER
FOR ALL THE BIRDS OF HEAVEN.

\bahasa
Hubungan manusia telah banyak berubah, dan telah berubah menjadi lebih buruk. Dalam semua dimensi hubungan yang lebih dalam telah hilang: istri bukan lagi seorang istri, tapi hanya pacar; suami bukan lagi seorang suami, tapi hanya pacar. Persahabatan itu bagus, tapi tidak dapat terlalu dalam. Pernikahan adalah sesuatu yang terjadi secara mendalam. Itu adalah komitmen secara mendalam, dan kecuali engkau berkomitmen pada dirimu, engkau tetap dangkal. Kecuali engkau berkomitmen pada diri sendiri, engkau tidak pernah mengambil lompatan. Engkau dapat mengapung dipermukaan, tapi kedalamannya bukan untukmu.

\english
Human relationships have changed a lot, and have changed for the worse. In all dimensions the deeper relationships have disappeared: the wife is no longer a wife, but just a girlfriend; the husband is no longer a husband, but just a boyfriend. Friendship is good, but cannot be very deep. Marriage is something which happens in depth. It is a commitment in depth, and unless you commit yourself you remain shallow. Unless you commit yourself you never take the jump. You can float on the surface, but the depths are not for you.

\bahasa
Tentu saja, untuk masuk ke kedalaman sangat berbahaya -- pasti begitu, karena di permukaan engkau sangat efisien. Di permukaan engkau dapat bekerja seperti robot; tidak ada kesadaran yang dibutuhkan Tapi engkau harus lebih waspada, semakin engkau menembus ke kedalaman, karena pada setiap saat kematian itu mungkin terjadi. Ketakutan akan kedalaman telah menciptakan kedangkalan dalam semua hubungan-hubungan. Hubungan-hubungan itu menjadi masa remaja.

\english
Of course, to go into the depths is dangerous -- bound to be so, because on the surface you are very efficient. On the surface you can work like an automaton; no awareness is needed. But you will have to be more and more alert, the more you penetrate into the depths, because at every moment death is possible. Fear of depth has created a shallowness in all relationships. They have become juvenile.


